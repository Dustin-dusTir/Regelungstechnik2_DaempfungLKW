% !TeX spellcheck = en_GB
% HOCHSCHULE REUTLINGEN
% TEMPLATE
% PROF. NOTHOLT

\documentclass[a4paper, 10pt]{IEEEtran}

%%% You can add the packages you may need
\usepackage[bookmarksopenlevel=1, colorlinks=true, allcolors=black, unicode]{hyperref}
\usepackage[cmyk, hyperref, table]{xcolor}
\usepackage{graphicx}


	\title{Guidelines for the final report on the project-based examination during CoViD-19 pandemic (SoSe2020)}% Der Titel
	\author{Prof. A. Notholt (13732)% Name und (Matrikel-Nr.)
			-- School of Engineering, Mechatronics Bachelor % Studiengang
			\thanks{Diese Hausarbeit wird in die eingereichte Fassung als Grundlage für die Benurteilung 
		            der Prüfungsleistung für die Prüfung
	            	60040099 % Prüfungsnummer
	            	Regelungstechnik/-systeme. %Prüfungsname
            	 	Der/Die Autor/Autorin versichert, dass dies sein/ihr eigenen Werk ist und alle entnommene Teile anderer Werke richtig zitiert sind.
             		}}
	\markboth%
		{Regelungstechnik und innovative Energiesysteme}% Bitte hier das Fach einfügen (Subject)
		{Prüfungsnummer 60040999}% Bitte hier deine Prüfungsnummer einfügen (aus dem HIP)


\begin{document}

	
	\maketitle
	
	\begin{abstract} %Hier kommt eine Zusammenfassung des Projektes
		This paper presents the general guidelines for the presentation of the project for the evaluation of the course. The source code can also be directly used as the format.
	\end{abstract}
	
	\section{Introduction}
	
	The purpose of this paper is to give our students a guideline on the format expected for the presentation of written reports for its evaluation in this summer semester 2020. It also includes the evaluation criteria for these projects.
	
	\section{Project targets}
	
	The aim of this project is to develop the capability of the student to communicate his/her work through the written word. To achieve this, the student must:
	
	\begin{itemize}
		\item Analyse and understand a problem in control engineering/renewable energy
		\item Develop a mathematical model of the behaviour of the process in question
		\item Define the required outcome
		\item Develop a suitable strategy adequate to the problem
		\item Test his/her strategy against the model
		\item Fine tune the result
		\item Critically address his/her observations and draw conclusions
	\end{itemize}
	
	\section{Theory}
	
	Scientific reports usually follow the scientific method as described in \cite{scientificMethod} as explained in the following sections.
	
	\subsection{Formulation of a question}
	
	In this step, a question is derived from an observation or curiosity. It is a question which usually asks for an explanation such as ``Which control strategy is the best to control the position of an elevator?'' The question and the background of this question is usually formulated in the \emph{Introduction} and/or ``Project targets''.
	
	\subsection{Hypothesis}
	
	A hypothesis is a conjecture to which you arrive by following logical and theoretical steps. The necessary theory involved in your questioning is written in the section \emph{Theory}.
	
	\subsection{Prediction}
	
	In this step you define the possible outcomes of your hypothesis and show how do you arrive to them. In engineering this would be the design process and is usually presented in the section regarding \emph{Methodology}.
	
	\subsection{Testing}
	
	Your hypothesis/design must be critically analysed and validated. For this you need to test it (by means e.g. of simulation). Your findings come in the section \emph{validation}.
	
	\subsection{Analysis}
	
	Through analysis of the results you may conclude if your hypothesis was correct or not. You should also explain what went wrong, under which circumstances the hypothesis would be right, etc. These findings come into the section \emph{Conclusions}.
	
	\section{Methodology} 
	
	In order to have a fast and homogeneous evaluation you should follow the rules laid out in the next subsections.
	
	\subsection{Structure}
	
	All reports must have the same structure at the \emph{section} level. Subsections may be adapted according to the specifics of the project. Table~\ref{tbl:sections} shows the required names in English and German.
	
	\begin{table}[hb]
		\caption{Required section headers in the project}\label{tbl:sections}
		\centering
		\begin{tabular}{cp{12em}p{12em}}
			\hline
			\bfseries Nr. & \bfseries English & \bfseries German\\
			\hline
			I & Introduction & Einführung\\
			II & Project targets & Projektziele \\
			III & Theory & Theorie \\
			IV & Methodology (Controller design) & Methodik (Reglerentwurf)\\
			V & Validation (Parameter tuning) & Validierung/Test \\
			VI & Observations and other effects & Beobachtungen und andere Effekte\\
			VII & Conclusions & Schlussfolgerungen\\
			\hline
		\end{tabular}
	\end{table}
	

	\subsection{Length}
	
	There is no minimum length, however, the report must include a comprehensible explanation of the different sections. The maximum length for the report is 8 pages.
	
	\begin{figure*}[hbt]
		\includegraphics[width=\textwidth]{escritorio}
		\caption{Prof. Notholt's home office set up as a double column.}\label{fig:doublecolumn}
	\end{figure*}
	
	\subsection{Design}
	
	\subsubsection{General design}
	
	Tue work must be presented as a two column article with a serif font (Computer modern, Times, etc.)  size of minimum 9\,pt and a maximum of 10\,pt. The margins should be within 1\,cm and 2\,cm.
	
	\subsubsection{Figures and Tables}
	Tables and figures must be captioned (Tables above, Figures below) and \emph{must be referenced in the document}. Both tables and figures may be presented in one column or in both columns as depicted in Figure~\ref{fig:doublecolumn}. Please do not include two-column tables or figures in the first page!
	

	
	\subsubsection{Formulae}
	Formulae shall be enumerated and can be referenced such as (\ref{eq:einstein}) if necessary.
	
	\begin{equation}
		E=m\cdot c^2 \label{eq:einstein}
	\end{equation}
	
	\subsubsection{References}
	References must be clearly and unequivocal stated and must be quoted. Inline quoting can be ``Important debates in the history of science concern rationalism''\cite{scientificMethod}. Block quoting can be:
	\begin{quotation}
		Important debates in the history of science concern rationalism, especially as advocated by René Descartes; inductivism and/or empiricism, as argued for by Francis Bacon, and rising to particular prominence with Isaac Newton and his followers; and hypothetico-deductivism, which came to the fore in the early 19th century.\cite{scientificMethod}
	\end{quotation}
	
	References may be used either via BibTeX of they could be written in the document as presented in this example. In any case, the references must include author, title, edition and pages for printed media or URL and last access for electronic media. Please remember that an electronic book is still a printed media.
	
	\section{Validation}
	
	The project report and presentation will account for 100\% of your grade. The marking scheme is based on a 100-point scale and is distributed through the criteria presented in Table~\ref{tbl:grading}.
	
	\begin{table*}[h]
		\caption{Grading scheme for projects in SoSe 2020}\label{tbl:grading}
		\centering
		\begin{tabular}{l p{0.13\linewidth} p{0.5\linewidth} c}
			\hline
			\bfseries \# & \bfseries Evaluation point & \bfseries Criteria & \bfseries Points \\
			\hline
			1 & Introduction (10) & Is present & 1 \\
			\cline{3-4}
			 & & Clearly states the problem & 6 \\
			 \cline{3-4}
			 & & Clearly states the relevance of the problem & 3\\
			 \hline
			 2 & Project targets (10) & Is present & 1 \\
			 \cline{3-4}
			 & & Problem is correctly defined & 6 \\
			\cline{3-4}
			 & & Expected outcomes are clearly stated & 3 \\
			 \hline
			 3 & Theory & Is present & 1 \\
			\cline{3-4}
			 & & The proposed model covers the important/relevant characteristics & 6 \\
			\cline{3-4}
			 & & The model is so described that it can be reproduced & 3\\
			 \hline
			 4 & Methodology & Is present & 1 \\
			\cline{3-4}
			 & & Method is clearly stated (e.g. which controller will be used and why) & 6 \\
			\cline{3-4}
			 & & Preliminary results (as input for simulation/validation) are clearly stated & 3\\
			 \hline
			 5 & Validation & Is present & 1 \\
			\cline{3-4}
			  & & Results are plausible & 3\\
			\cline{3-4}
			  & & Results are discussed and potential limitations of the model are explained & 6\\
			 \hline
			 6 & Observations & Is present & 1\\
			\cline{3-4}
			& & Relevant effects are discussed & 3\\
			\cline{3-4}
			& & Explanations to the observed effects are relevant & 6\\
			\hline
			7 & Conclusions & Is present & 1 \\
			\cline{3-4}
			& & Conclusions address hypothesis and project aims and goals & 9 \\
			\hline
			8 & Final presentation & Sticks to 5 Minute maximum & 5 \\
			\cline{3-4}
			&& Presents the content professionally & 5\\
			\cline{3-4}
			&& Addresses all sections of the report & 10 \\
			\cline{3-4}
			&& Public can understand about the project, solution and effects even if they have little experience on the specific topic & 10\\
			\hline
			\hline
			 & Total & & 100 \\
		\end{tabular}
	\end{table*}

	\subsection{Points grading scheme}
	
	In all categories, there is the possibility of having either one, three, six or nine points. One-point rubrics are graded on present/not present. If the section is not present, the student will lose the point.
	
	Three-, six- and nine-point rubrics are graded as follows:
	
	\begin{enumerate}
		\item One third of the points are awarded if the grading rubric represents the minimum effort necessary to comply with it (read and quote).
		\item Two thirds of the points are awarded if the student is able to explain the required information  (interpret and apply).
		\item The full points are awarded if the student is able to interpret the information, apply it and discuss it.
	\end{enumerate}

	{\bfseries Example:} Describe the transfer function of an RC filter (input to output voltage) for 6 points
	
	\begin{description}
		\item[0p] $U_{C} = \frac{1}{C}\int i_{c} \cdot dt$
		\item[2p] Prof. Notholt says in script: $G(s) = \frac{1}{RCs+1}$
		\item[4p] ``The transfer function is the quotient of the Laplace input signal to the Laplace output signal'' [Lunze 1999] thus being $G(s) = \frac{1}{Ts+1}$ with $T=RC$
		\item[6p] The RC filter can be considered like a complex voltage divider with $Z = 1/sC$, the solution is then $U_{C} = \frac{1/sC}{R + 1/sC} \cdot U_\textrm{e}$ or, simplified and as transfer function: $G(s) = U_{C}/U_\textrm{e} = \frac{1}{RCs + 1}$ 
	\end{description}
	
	
	\section{Conclusions}
	
	This paper has described the minimum requirements for the written project and final presentation. Please meake sure you read this document completely and follow the guidelines. Prof. Notholt wish you the best for your endeavor!

	\begin{thebibliography}{1em}
		\bibitem{scientificMethod} Wikipedia, ``The scientific method'', online, \url{https://en.wikipedia.org/wiki/Scientific_method}, accessed 20.4.2020.
	\end{thebibliography}
	


\end{document}